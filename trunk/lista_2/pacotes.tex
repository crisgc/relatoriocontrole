% Margens ----------------------------------------------------------------------
\usepackage[top=3cm,left=3cm,right=2cm,bottom=2cm]{geometry}

% Pacotes Principais -----------------------------------------------------------
\usepackage[portuges,brazil]{babel}
\usepackage[utf8]{inputenc}
\usepackage[T1]{fontenc} 
\usepackage{ae}
\usepackage{indentfirst} % Indenta os primeiros parágrafos

% Fórmulas
\usepackage{icomma}
\usepackage{cancel}

% Formatação
\usepackage{url}

% Figuras e Imagens ------------------------------------------------------------
\usepackage{graphicx}
% Figuras lado a lado
\usepackage{epsfig}
\usepackage{subfigure}
\usepackage{color}

% Utilizar H para inserir as imagens REALMENTE onde eu desejo
\usepackage{float}

% Fontes -----------------------------------------------------------------------
\usepackage[T1]{fontenc}
\usepackage{pslatex}

% Simbolos ---------------------------------------------------------------------
\usepackage{textcomp}
\usepackage{amsmath}

% Tabelas ----------------------------------------------------------------------
%\usepackage{multicol}
\usepackage{multirow}

% Codigos ----------------------------------------------------------------------
% Comentários em bloco
\usepackage{verbatim}
\usepackage{listings}

% Biliografia ------------------------------------------------------------------
\usepackage{harvard}
